\documentclass{article}

\usepackage[utf8]{inputenc}
\usepackage{blindtext}
\usepackage{graphicx}
\usepackage{float} % Elimina la condicion de flotante de las imagenes
\usepackage[english, activeacute]{babel}
\usepackage{amsmath}
\usepackage{etoolbox,fancyhdr,xcolor} % Colores
\usepackage[hidelinks]{hyperref} % Indice dinámico
\usepackage[a4paper]{geometry}
\usepackage{listings} % Para mostrar fragmentos de código.
\usepackage{multicol}
\usepackage[lighttt]{lmodern}
\usepackage{tikz}

\usetikzlibrary{shapes.geometric, arrows}

\title{Assignment 3.}
\author{José Manuel Izquierdo Ramírez}

\newcommand{\headrulecolor}[1]{\patchcmd{\headrule}{\hrule}{\color{#1}\hrule}{}{}}
\newcommand{\footrulecolor}[1]{\patchcmd{\footrule}{\hrule}{\color{#1}\hrule}{}{}}
\pagestyle{fancy}
\fancyhf{}% Clear header/footer
\fancyhead[L]{\textsl{\leftmark}}
\fancyfoot[C]{\thepage}% \fancyfoot[R]{\thepage}
\renewcommand{\headrulewidth}{0.4pt}% Default \headrulewidth is 0.4pt
\renewcommand{\footrulewidth}{0.4pt}% Default \footrulewidth is 0pt
\headrulecolor{cyan!70}% Set header rule colour to 70% red.
\footrulecolor{cyan!70}
\renewcommand{\sectionmark}[1]{\markboth{#1}{}}
\renewcommand{\subsectionmark}[1]{\markright{#1}}

% Multicol
\setlength{\columnseprule}{0.4pt}
\def\columnseprulecolor{\color{cyan!70}}

% Anular el sangrado
%\setlength{\parindent}{0cm}


\lstset{basicstyle=\ttfamily, keywordstyle=\bfseries}
\ttfamily
\DeclareFontShape{OT1}{lmtt}{m}{it}
    {<->sub*lmtt/m/sl}{}

\begin{document}

    \tikzstyle{startstop} = [rectangle, rounded corners, minimum width = 2cm, minimum height=1cm,text centered, draw = black]
    \tikzstyle{io} = [trapezium, trapezium left angle=70, trapezium right angle=110, minimum width=2cm, minimum height=1cm, text centered, draw=black]
    \tikzstyle{process} = [rectangle, minimum width=3cm, minimum height=1cm, text centered, draw=black]
    \tikzstyle{decision} = [diamond, aspect = 3, text centered, draw=black]
     % Forma de flecha
    \tikzstyle{arrow} = [->,>=stealth]

    \begin{titlepage}
        
        \centering
        {\LARGE\bfseries Assignment 3. \par}
        \vspace{0,5cm}
        {\itshape\Large Metaheuristics \par}
        \vspace{0,5cm}        
        \vspace{1cm}
        \includegraphics[width=0.6\textwidth]{../media/Logo_UCO.png}\par
        \vspace{3cm}
        {\LARGE\bfseries José Manuel Izquierdo Ramírez \par}

        %{\large \today\par}

    \end{titlepage}
    
    \begin{index}
        \tableofcontents
        \newpage
        \listoffigures
    \end{index}

    \newpage
    
    \section{Introduction}
    
    In this practice we have been asked to implement an algorithm of our choice to solve the problem given.

    \section{Problem approach}

    With a given dataset we have to implement an algorithm that finds patterns that repeat throughout it.

    \subsection{Decide on the encoding}

    At first I though of using a tree-based encoding, but after some deliberation I decided to use an alphabetic encoding.

    \begin{lstlisting}[language=Python]
        
        solution = ['R', 'D', 'H', 'R', 'B', 'F', 'A']
        
        solution = ['R', 'D', ['H', 'J'], 'R', 'B', 'F']
        
    \end{lstlisting}

    When creating random solutions I allow that two events can happen at the same time with an arbitrarily set probability. 

    \begin{lstlisting}[language=Python]
        ...
        for i in range(nSolutions):
        solutions = []
        
        #Probability of having more than one event at the same time

        prob_occur_same_time = 0.9

        it = 0

        #Generate solutions between the min and max length randomly
        while it < random.randint(min_length,max_length):
            
            #It generates solutions with events that occur at the 
            same time with a probability of 90%.

            if random.randint(1,100) <= (prob_occur_same_time*100):
                
                ...
            else:
               
                ...            
            it += 1

        population.append([solutions,evaluateSolution(solutions,
                            events, list_of_dictionaries, 
                                len(solutions) )])

    \end{lstlisting}

    \subsection{Decide on the heuristic}

    To start with I really liked the idea of using artificial ant colony algorithm to solve it, 
    but as much as I thought about it I didn't see that it fit completely with the problem we are handling.

    Finally I decided to implement a genetic algorithm; having done it in practice 2 I thought of adding 
    something more, that is why I have made a few implementations that I will detail later on.

    \subsection{How to evaluate the solutions?}

    In my opinion the evaluation function is one of the most important parts of a genetic algorithm, 
    because if it is not accurate the population will not evolve correctly, will not converge, and 
    will make it difficult to reach a global or even a local maximum.

    To do this I start by using an auxiliary function \textit{readFile()}, which returns a list of dictionaries, 
    where each dictionary has as key the events and as value a list of the times in which the event has happened.
    \begin{figure}[H]
        \begin{tikzpicture}[node distance=1cm]
            % Definir la forma específica del diagrama de flujo
            \node[startstop](start){Start};
            \node[process, below of = start, yshift = -0.5cm](pro0){For dictionary in list of dictionaries};
            \node[io, below of = pro0, yshift = -0.5cm](in1){Initialize last epoch, count, diff letters and iterations};

            \node[process, below of = in1, yshift = -0.5cm](pro1){For element in solution};

            \node[decision, below of = pro1, yshift = -1cm](dec0){if type(element) == list};
            \node[process, right of = dec0, xshift = 2cm, yshift = -2cm](pro2){dictionary.get(element[0]) and dictionary.get(element[1])};
            \node[process, left of = dec0, xshift = -3cm, yshift = -2cm](pro3){dictionary.get(element)};
        
            \node[decision, below of = pro2, yshift = -1cm](dec1){epoch1 == epoch2};
            \node[process, right of = dec1, xshift = 2cm, yshift = -3.5cm](pro4){Penaliza};
        
            \node[decision, left of = dec1, xshift = -2cm, yshift = -2cm](dec2){epoch $\geq$ last epoch};
            \node[process, left of = dec2, xshift = -2cm, yshift = -1.5cm](pro5){Puntua};
            \coordinate (converge) at (0cm, -16cm);
            \node[io, below of = converge, yshift = -0.5cm](out){return total/length, frequency, diff letters};
        
            \coordinate (left) at (-6cm, -16cm);
            \coordinate (elementloop) at (0cm, -5.2cm);
            \coordinate (left2) at (-8.2, -16cm);
            \coordinate (dicionaryloop) at (0cm, -2.2cm);

            % Conectar forma específica
            \draw [arrow] (start) -- (pro0);
            \draw [arrow] (pro0) -- (in1);
            \draw [arrow] (in1) -- (pro1);
            \draw [arrow] (pro1) -- (dec0);
            \draw (dec0) -| node [left] {N} (pro3);  
            \draw (dec0) -| node [right] {Y} (pro2);  
            \draw [arrow] (pro2) -- (dec1);   
            \draw (dec1) -| node [left] {Y} (dec2);  
            \draw (dec1) -| node [right] {N} (pro4);
            \draw [arrow] (dec2) -| node [left] {Y} (pro5);
            \draw [arrow] (dec2) -| node [right] {N} (pro4);
            \draw [arrow] (pro3) -- (dec2);

            \draw [arrow] (pro5) -- (converge);
            \draw [arrow] (pro4) -- (converge);
            \draw [arrow] (converge) -- (out);
        
            \draw [arrow] (converge) -- (left);
            \draw [arrow] (left) |- (elementloop);

            \draw [arrow] (converge) -- (left2);
            \draw [arrow] (left2) |- (dicionaryloop);
        
        \end{tikzpicture}

        \caption{Class Diagram of Evaluation Function}
        \label{Class Diagram of Evaluation Function}
    \end{figure}

    \section{Implementation}

    I started by doing a basic genetic algorithm, but as I said before 
    I wanted to do something different from the implementation done in assignment 2.

    One of the biggest problems of this algorithm is the computational time needed by the evaluation function,
    so reducing the number of calls it recieves is very worth it. I tried to parallelise the evaluation of the solutions, 
    but the result I got was a longer evaluation time. 

    Also to have more intensification I thought on applying Simulated Annealing to change the degree to which worse solutions are accepted
    or, what finally I applyed, Hill Climbing to search the best neighbour of good enough solutions.

    \subsection{Parallelism}

    This was my first time working with python threads, and I think this was one of the reasons why the result is so bad.

    I had thought about testing this implementation but the time it takes makes it prohibitive, we would be talking about hours.

    \begin{lstlisting}
    def geneticAlgorithmMultiprocessing(nSolutions,
        maxGenerations,mProb,cProb,k,elitism, 
                min_length, max_length):

        ...
        while it < maxGenerations:

            nSolutions = applyGeneticOperator(population,k,
                                                    cProb,mProb,
                                                        events)

            #Generational model
            population = []

            process = []
            for solution in nSolutions:
            
                # initialize total and frequency
                total = multiprocessing.Value('d',0.0)
                frequency = multiprocessing.Value('d',0.0)
                diff_letters = multiprocessing.Value('d',0.0)
                # creating a lock object
                #lock = multiprocessing.Lock()

                # creating new processes
                p = multiprocessing.Process(
                    target=evaluateSolution_thread,
                        args=(solution[0],events,
                            list_of_dictionaries,len(solution[0]),
                            total, frequency, diff_letters))

                process.append(p)

                # starting processes
                p.start()

                # wait until processes are finished
                for p in process:
                    p.join()

                population.append([solution[0], (total.value, 
                    frequency.value, diff_letters.value)])

            it+=1
        ...
        
    \end{lstlisting}

    \subsection{Applying Caching}

        By applying caching we can see how the time and computational cost is practically halved.
        
        \begin{lstlisting}
        def geneticAlgorithmMultiprocessing(nSolutions,
            maxGenerations,mProb,cProb,k,elitism, min_length, 
                max_length):
    
            ...
            while it < maxGenerations:
    
                nSolutions = applyGeneticOperator(population,k,
                    cProb,mProb,events)
    
                #Generational model
                population = []
    
                process = []
                for solution in nSolutions:
                
                    if solution in evaluated_solutions:
                        index = 
                            evaluated_solutions.index(solution)
                        population.append(
                            evaluated_solutions[int(index)])
                    
                    else:
                        population.append(
                            [solution[0],evaluateSolution(
                                solution[0],events,
                                    list_of_dictionaries,
                                        len(solution[0]) )])
    
            it+=1
            ...
            
        \end{lstlisting}

    \subsection{Applying Hill Climbing}

    Implementing hill climbing has pros and cons.
    On one hand, the population converges much earlier, resulting in better results in fewer generations.
    On the other hand, the number of evaluations the algorithm performs while searching for the best neighbour makes it slower.

    \begin{lstlisting}
        def getBestNeighbor(solution, data, events, 
            list_of_dictionaries):
            
            ##Get the neighbors
            neighbors = []
            l=len(solution[0])

            for i in range(l):
                for j in range(i+1, l):
                    n = solution[0].copy()
                    n[i] = solution[0][j]
                    n[j] = solution[0][i]
                    neighbors.append(n)
            
            ##Get the best neighbor
            bestNeighbor = neighbors[0]
            ##evaluateSolution(solutions,events,
                list_of_dictionaries,
                    len(solutions) )
            
            bestTotal,bestFrequency,bestNEvents = 
                evaluateSolution(bestNeighbor, events,
                 list_of_dictionaries, len(bestNeighbor))
    
            for neighbor in neighbors:
        
                Total,Frequency,NEvents = 
                    evaluateSolution(neighbor, events, 
                        list_of_dictionaries, len(bestNeighbor))
        
                if Total > bestTotal:
                    bestTotal = Total
                    bestFrequency = Frequency
                    bestNEvents = NEvents

                    bestNeighbor = neighbor

            return bestNeighbor, 
                (bestTotal, bestFrequency, bestNEvents)

        def hillClimbing(data, events, list_of_dictionaries):
    
            solution = []
    
            solution.append(data[0][0])

            total = data[0][1][0]
    
            neighbour = getBestNeighbor(solution, data, 
                events, list_of_dictionaries)

            while neighbour[1][0] > total:
                solution = neighbour
                total = neighbour[1][0]
                neighbour = getBestNeighbor(solution, data, 
                    events, list_of_dictionaries)

            return solution[0]
        
        def geneticAlgorithmMultiprocessing(nSolutions,
            maxGenerations,mProb,cProb,k,elitism, 
                min_length, max_length):
    
                ...
                while it < maxGenerations:
    
                    ...

                    population.sort(reverse=True,key=lambda 
                        population: (population[1][0], 
                            population[1][2]) )

                    if population[0][1][0] > 60:
                        bestNeighbour = list(hillClimbing(
                            population, events, 
                                list_of_dictionaries))
            
                        population.pop()
                
                        population.append(
                            [bestNeighbour,evaluateSolution(
                                bestNeighbour,events,
                                    list_of_dictionaries,
                                        len(bestNeighbour) )])
            
            it+=1
            ...
    \end{lstlisting}

\end{document}