\documentclass{article}

\usepackage[utf8]{inputenc}
\usepackage{blindtext}
\usepackage{graphicx}
\usepackage{float} % Elimina la condicion de flotante de las imagenes
\usepackage[english, activeacute]{babel}
\usepackage{amsmath}
\usepackage{etoolbox,fancyhdr,xcolor} % Colores
\usepackage[hidelinks]{hyperref} % Indice dinámico
\usepackage[a4paper]{geometry}
\usepackage{listings} % Para mostrar fragmentos de código.
\usepackage{multicol}

\title{Assignment 1. Hill climbing and simulated annealing }
\author{José Manuel Izquierdo Ramírez}

\newcommand{\headrulecolor}[1]{\patchcmd{\headrule}{\hrule}{\color{#1}\hrule}{}{}}
\newcommand{\footrulecolor}[1]{\patchcmd{\footrule}{\hrule}{\color{#1}\hrule}{}{}}
\pagestyle{fancy}
\fancyhf{}% Clear header/footer
\fancyhead[L]{\textsl{\leftmark}}
\fancyfoot[C]{\thepage}% \fancyfoot[R]{\thepage}
\renewcommand{\headrulewidth}{0.4pt}% Default \headrulewidth is 0.4pt
\renewcommand{\footrulewidth}{0.4pt}% Default \footrulewidth is 0pt
\headrulecolor{cyan!70}% Set header rule colour to 70% red.
\footrulecolor{cyan!70}
\renewcommand{\sectionmark}[1]{\markboth{#1}{}}
\renewcommand{\subsectionmark}[1]{\markright{#1}}

% Multicol
\setlength{\columnseprule}{0.4pt}
\def\columnseprulecolor{\color{cyan!70}}

% Anular el sangrado
\setlength{\parindent}{0cm}



\begin{document}

    \begin{titlepage}
        
        \centering
        {\LARGE\bfseries Assignment 1. Hill climbing and simulated annealing \par}
        \vspace{0,5cm}
        {\itshape\Large Metaheuristics \par}
        \vspace{0,5cm}        
        \vspace{1cm}
        \includegraphics[width=0.6\textwidth]{../media/Logo_UCO.png}\par
        \vspace{3cm}
        {\LARGE\bfseries José Manuel Izquierdo Ramírez \par}

        %{\large \today\par}

    \end{titlepage}
    
    \begin{index}
        \tableofcontents
        \newpage
        \listoffigures
    \end{index}

    \newpage
    
    \section{Exercise 1.}
    \subsection{How does this algorithm behave as we increase the complexity of the problem (number of cities in the TSP)?}

        1. behaviour algorithm

            pop = 100
            #gen = 1000
            cp = 0.8
            mp = 0.2

            Obtener la probabilidad de obtener el mejor incrementando la cantidad de objetos

            Para obtner esto cogemos el mejor resultado o el mejor resultado y los cercanos a este tambn.

            Best value +- 10%

            2. Do the parameter setup on a problem that is very harder.

            Population: 100,150,200

            #Gen: 1000, 2000, 3000

            cp: 0.7 0.8 0.9

            mp: 0.1 0.2 0.3

            3. Elitism

            with the same parameters compare.
\end{document}